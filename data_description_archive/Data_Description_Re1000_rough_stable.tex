\documentclass[11pt]{article}
\usepackage{graphicx}
\usepackage[top=10mm, bottom=25mm, left=30mm, right=30mm, includehead]{geometry}
\usepackage{fancyhdr}
\usepackage{hyperref}
\usepackage{mathptmx}
\usepackage{amssymb}
\usepackage{tikz}
\usepackage{longtable}
\usepackage{lipsum}
\usepackage{float}
\usepackage{pdfpages}
\usepackage{sectsty}
\pagestyle{fancy}
\usepackage{lmodern}
\usepackage{enumitem}
\usepackage{multicol} 
\usepackage{amsmath}
\usepackage{booktabs}
\usepackage{colortbl}

\fancyhf{}
\date{\nodate}
\setlength{\headheight}{23.10004pt}
 
\fancyhead[R]{REFUBIUM - FREIE UNIVERSITÄT BERLIN}

% -------------------------------------------------------------
% ------------------ start your abstract here -----------------
% -------------------------------------------------------------

\newcommand{\dstitle}{Turbulent Ekman flow with cubic small-scale surface roughness under stable stratification \\ ($Re_D=1000$, $Ri_\Lambda=[0,\dots,256]$)}
\newcommand{\subtitle}{Direct numerical simulation -- Set-up and vertical profiles}
\newcommand{\doi}{10.17169/refubium-45292}
\newcommand{\collectionTitle}{Turbulent boundary layers simulated by resolved-turbulence simulation}
\newcommand{\collectionLink}{refubium.fu-berlin.de/handle/fub188/42710}
\newcommand{\tlabLink}{github.com/turbulencia/tlab}
\newcommand{\tlabName}{\textcolor{gray}{\texttt{tLab}}}
\newcommand{\correspondence}{j.kostelecky@posteo.de; cedrick@posteo.de}
\newcommand{\authors}{
    Jonathan Kostelecky \footnote{\href{mailto:j.kostelecky@posteo.de}{j.kostelecky@posteo.de}}$^{\star,\blacktriangle}$, \hspace{0.05cm}  
    Cedrick Ansorge     \footnote{\href{mailto:cedrick@posteo.de}{cedrick@posteo.de}}$^\star$ \\  
    {\small \light{$^\star$ \emph{ Freie Universit\"at Berlin, Institut f\"ur Meteorologie} \\
    $^{\blacktriangle}$ \emph{Universit\"at zu K\"oln, Institut f\"ur Geophysik und Meteorologie}}}}
\newcommand{\printdoi}[1]{\href{https://dx.doi.org/#1}{#1}}
\newcommand{\light}[1]{\textcolor{black!80}{#1}}
\allsectionsfont{\bfseries\sffamily} 
\hypersetup{
    colorlinks = true,
    urlcolor   = blue,
    linkcolor  = blue,
    citecolor  = black
}
%-----------------------------------------------------------------------------------------------------------------------------------
%-----------------------------------------------------------------------------------------------------------------------------------
\begin{document}
%-----------------------------------------------------------------------------------------------------------------------------------
%-----------------------------------------------------------------------------------------------------------------------------------

{\fontfamily{lmss}\selectfont
\begin{centering}
    \light{\large Dataset description} \\[1em] 
    {\LARGE \bfseries \sffamily\dstitle} \\[1em]
    \light{\large\subtitle} \\[1em]
    \authors \\ 
\end{centering}
\light{\hfill\today}}

%-----------------------------------------------------------------------------------------------------------------------------------
\section{Metadata}
%-----------------------------------------------------------------------------------------------------------------------------------

\begin{description}

    \sffamily

\item[\textcopyright] This work is licensed under the \href{https://creativecommons.org/licenses/by/4.0}{creative commons CC BY 4.0 license}. \\ 
{\small\light{You must give appropriate credit, provide a link to the license, and indicate if changes were made. You may do so in any reasonable manner, but not in any way that suggests the licensor endorses you or your use. }}

\item[\sffamily Collection] This dataset is part of the collection \href{https://refubium.fu-berlin.de/handle/fub188/42710}{\emph{Turbulent wall-bounded flow}\footnote{\url{refubium.fu-berlin.de/handle/fub188/42710}}}. \\ 
{\small \light{The collection is freely available and hosted by Refubium, the  institutional repository of Freie Universit\"at Berlin.}}

\item[\sffamily DOI]\printdoi{\doi}

\item[\sffamily HPC systems] The data was generated under the project \texttt{TrainABL} on the supercomputer \texttt{HAWK} at Höchstleistungsrechenzentrum Stuttgart (HLRS, in Germany). 

\item[\sffamily Code] The data was generated by the tool-suite for turbulence simulation \href{https://github.com/turbulencia/tlab}{\tlabName\footnote{\url{github.com/turbulencia/tlab}}}.

\item[\sffamily Related Publication] \phantom{A}~\\
    \printdoi{10.1017/jfm.2024.542}\\

\end{description}

%-----------------------------------------------------------------------------------------------------------------------------------
\section{The dataset}
%-----------------------------------------------------------------------------------------------------------------------------------
\subsection{Physical setting}
%-----------------------------------------------------------------------------------------------------------------------------------
% N, S001, S002, S004, S005, S008, S012, S016, S020, S032, S042, S064, S128, S128P, S192P, S256P
The physical case corresponds to stratified Ekman flow over a rough surface. The canonical flow configuration is characterized by four parameters: the geostrophic wind vector \linebreak $\mathbf{G}=\left(G_1,G_2,0\right)^T$ (with the magnitude $G=\sqrt{G_1^2+G_2^2}$, rotated here by $\approx18.1^\circ$ w.r.t. the coordinate direction $O_x$), the constant kinematic fluid viscosity $\nu$, the Coriolis parameter $f$, and the buoyancy difference $B_0$ between the wall and free stream. The Rossby radius $\Lambda=G/f$ is the length scale implied for this choice of parameters. 
%
The flow is then governed by two dimensionless numbers, since the molecular Prandtl number $Pr=1$, i.e. the kinematic diffusivity equals the viscosity, and the Rossby number $Ro=1$. The Reynolds number $Re$ and the Froude number $Fr$ are given as (for comparison, the scaled bulk Richardson number $Ri_B$)
% 
\begin{equation}
    Re_\Lambda = \frac{G \Lambda}{\nu},\quad
    Fr_\Lambda=\frac{G^2}{B_0 \Lambda}=Ri_\Lambda^{-1}\quad \mathrm{and} \quad
    Ri_{B} = \frac{B_0 \delta_\mathrm{neutral}}{G^2}.
\end{equation}
%
The small-scale surface roughness at the lower domain boundary is given in the file \linebreak \texttt{geometry2d.nc} and is identical for all cases: $56 \times 56$ square blocks with a uniform height and width distribution. It also features identical statistical properties as case \texttt{r3} in the study \linebreak \printdoi{10.1017/jfm.2024.542}, but corresponds to a realization on a grid with slightly higher resolution\,/\,smaller grid spacing.
%
\subsection{Simulation cases}
%
\begin{table}
    \newcommand{\rc}{\rowcolor{blue!20}}
    \centering
    \begin{tabular}{lrcccccr}
        Case (\texttt{ID}) & $Ri_\Lambda$ & $Ri_B$ & Grid &  $t^-_\text{start} [1/f]$ &  $t^-_\text{end} [1/f]$ &  $\Delta t^-_\text{sim} [1/f]$ &  \#iterations \\ \bottomrule
        ~\vspace{-.75em}\\
        \texttt{N    } &   0 & \hphantom{1}0.000 & A & 64.96 & 66.02 & 1.06 & 29500  \\ \bottomrule[0.25pt] 
        ~\vspace{-.75em}\\
        \texttt{S001 } &   1 & \hphantom{1}0.073 & A & 66.02 & 66.61 & 0.58 & 16000  \\
        \rc\texttt{S002 } &   2 & \hphantom{1}0.147 & A & 66.61 & 67.06 & 0.44 & 11800  \\
        \texttt{S004 } &   4 & \hphantom{1}0.293 & A & 67.06 & 67.78 & 0.71 & 17800  \\
        \rc\texttt{S005 } &   5 & \hphantom{1}0.367 & A & 67.78 & 68.26 & 0.48 & 11800  \\
        \texttt{S008 } &   8 & \hphantom{1}0.587 & A & 68.26 & 69.02 & 0.74 & 17800  \\
        \rc\texttt{S012 } &  12 & \hphantom{1}0.880 & A & 69.02 & 69.53 & 0.51 & 11800  \\
        \texttt{S016 } &  16 & \hphantom{1}1.174 & A & 69.53 & 69.98 & 0.44 & 10300  \\
        \rc\texttt{S020 } &  20 & \hphantom{1}1.467 & A & 69.98 & 70.38 & 0.39 & 8800  \\
        \texttt{S032 } &  32 & \hphantom{1}2.348 & A & 70.38 & 70.84 & 0.45 & 9800  \\
        \rc\texttt{S042 } &  42 & \hphantom{1}3.081 & A & 70.84 & 71.40 & 0.55 & 11800  \\
        \texttt{S064 } &  64 & \hphantom{1}4.695 & A & 71.40 & 71.84 & 0.43 & 8800  \\ 
        \rc\texttt{S128 } & 128 & \hphantom{1}9.390 & A & 71.84 & 72.32 & 0.48 & 8800 \\
        ~\vspace{-.75em}\\
        \bottomrule[0.25pt]
        ~\vspace{-.75em}\\
        \texttt{S128P} & 128 & \hphantom{1}9.390 & B & 72.32 & 74.27 & 1.95 & 40600  \\
        \rc\texttt{S192P} & 192 & 14.086            & B & 73.49 & 74.46 & 0.97 & 19000  \\
        \texttt{S256P} & 256 & 18.781            & B & 73.49 & 75.68 & 2.19 & 41400  \\
    \end{tabular}
    \caption{Simulations cases of this dataset. Time in eddy-turnover times $f^{-1}$; $t^-_\text{start}$ start, $t^-_\text{end}$ end and $\Delta t^-_\text{sim}$ total simulation time of the cases. $\#\text{iterations}$ is the total number of Runge--Kutta time-integration steps used for integration of the problem over the respective time span.}
    \label{tab:cases}
\end{table} 

This dataset contains 16 simulation cases which are driven by the same large-scale forcing and exposed to a rough surface. 
%
Members of the parametric set of simulations differ by stable density stratification, which is imposed via a Dirichlet boundary condition (viz. constant temperature difference between upper and lower domain boundary). The cases are listed in Tab.~\ref{tab:cases} and labelled by their \texttt{ID} (\texttt{N, S001,\dots, S128, \dots, S256P}) according to the stratification measured by the Richardson number $Ri_\Lambda$; \texttt{N} stands for neutral stratification, \texttt{S} for stable. The simulations were carried out sequentially in time starting from the neutral case \texttt{N}, a suffix \texttt{P} indicates three concurrent runs for very stable stratification (\texttt{S128P, S192P, S256P}).
\\
%
Here the bulk Richardson number $Ri_B$ is evaluated based on the boundary layer thickness of the neutrally stratified case \texttt{N} such that it does not evolve over the course of simulation. The choice of $Re_\Lambda=5\cdot10^5$ corresponds to $Re_D=1000$ ($Re_D=DG/\nu$, with the laminar Ekman-layer depth $D=\sqrt{2\nu/ f}$) and a friction Reynolds number $Re_{\tau} \approx 2700$ for the neutrally stratified case.  
%
Two computational grids are used: (A) $N_{xz}\times N_y=3840^2\times704$, (B) $3840^2\times576$ with similar spatial resolution of approximately $\Delta \left(xz\right)^{+}_{N}\times\Delta y_{N,\text{wall}}^+=2.6^2\times1.0$ wall units and a domain size in terms of the Rossby radius of $(L_{xz}\times L_y)/\Lambda^3=0.27^2\times0.26$ for grid (A), and~$0.27^2\times0.11$ for grid (B). 
%

\subsection{Contents of the dataset }
%-----------------------------------------------------------------------------------------------------------------------------------
This dataset holds the two metadata files 
\begin{itemize} 
    \item \texttt{Data\_Description\_Re1000\_rough\_stable.pdf} (this file) containing the data set description in portable document format (PDF), 
    \item \texttt{geometry2d.nc}, which holds the geometry of all cases in the form of a two-dimensional horizontal plane where the height of obstacles in grid cells from the ground level is given. 
\end{itemize} 
For the set of 16 simulations, the primary data are given in namelist files in ASCII format (denoted by suffix \texttt{.ini}) as required by the tool suite \texttt{tLab} (for details, the reader is referred to the documentation of the open-source code available under \href{https://github.com/turbulencia/tlab}{\texttt{github.com/turbulencia/tlab}}). The actual statistical data are provided in the network common data format self-documenting file type (netCDF) and denoted by the suffix \texttt{.nc}. The naming convention for the set of 16 simulations is as follows: 
\begin{itemize}
\item[] $\texttt{ri} \langle \text{ri}\rangle \texttt{\_}
\texttt{re} \langle \text{re}\rangle \texttt{\_}
\langle nx\rangle \texttt{x} \langle ny \rangle \texttt{x} \langle nz\rangle  \texttt{\_}
\langle \text{date}\rangle\texttt{\_}
\langle \text{case}\rangle\texttt{\_}
\langle \text{type}\rangle\texttt{.}
\langle \text{suffix}\rangle$
\end{itemize}
For example \texttt{ri00.00\_re1000\_3840x0704x3840\_20231206\_n\_avg.nc}, where

\begin{itemize}[itemindent=2em] 
    \item[$\langle \text{ri}\rangle$] is the Richardson number $Ri_\Lambda = G \Lambda / B_0$, 
    \item[$\langle \text{re}\rangle$] the Reynolds number $Re_D=G D /\nu$ with $D=\sqrt{2\nu/f}$, 
the Ekman layer depth scale 
    \item[$\langle \text{nx}\rangle$] the number of grid points in direction of Ox (similar for $\langle \text{ny}\rangle$, $\langle\text{nz}\rangle$)
    \item[$\langle \text{date}\rangle$] the start date of the simulation on the HPC cluster in the format YYYYMMDD. 
    \item[$\langle \text{case}\rangle$] the case identifier used in the corresponding paper, indicating the bulk stability. 
    \item[$\langle \text{type}\rangle$] the type of data, either \texttt{avg1s} for scalar statistics, \texttt{avg} for flow statistics, and 
    \item[$\langle \text{suffix}\rangle$] is the file suffix indicating the file type (\texttt{.ini} for primary data / \texttt{.nc} for statistics)
\end{itemize}
%-----------------------------------------------------------------------------------------------------------------------------------

%-----------------------------------------------------------------------------------------------------------------------------------
\subsection{Variable information} 
%-----------------------------------------------------------------------------------------------------------------------------------
The statistical data is available in self-documented netCDF format, and it contains a wide array of parameters, encompassing vertical profiles of velocity and scalar variables (temperature/buoyancy as active and for some cases also passive scalars), scalar and momentum budget terms, as well as statistical moments up to the fourth order of velocities, scalars, and derivatives. These parameters provide a comprehensive perspective on Ekman flow dynamics. They are organized into distinct groups. Within the subsequent table, you will find numerous variables grouped together, accompanied by their descriptions and associated equations. \\
In order to fully describe the geometry of the surface roughness, there is a horizontal domain slice in netCDF format, that describe the positions and heights of the roughness elements in grid points.
\par
%-----------------------------------------------------------------------------------------------------------------------------------

%-----------------------------------------------------------------------------------------------------------------------------------
%-----------------------------------------------------------------------------------------------------------------------------------
\end{document}
%-----------------------------------------------------------------------------------------------------------------------------------
%-----------------------------------------------------------------------------------------------------------------------------------