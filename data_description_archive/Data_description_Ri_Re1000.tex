\documentclass[11pt]{article}
\usepackage{graphicx}
\usepackage[top=10mm, bottom=25mm, left=30mm, right=30mm, includehead]{geometry}
\usepackage{fancyhdr}
\usepackage{hyperref}
\usepackage{mathptmx}
\usepackage{tikz}
\usepackage{longtable}
\usepackage{amsmath}
\usepackage{lipsum}
\usepackage{float}
\usepackage{pdfpages}
\usepackage{sectsty}
\pagestyle{fancy}
\usepackage{lmodern}
\usepackage{booktabs}
\usepackage{enumitem}
\usepackage{multicol}
\fancyhf{}
\date{\nodate}
\setlength{\headheight}{23.10004pt}

\fancyhead[R]{REFUBIUM - FREIE UNIVERSITÄT BERLIN}

% -------------------------------------------------------------
% ------------------ start your abstract here -----------------
% -------------------------------------------------------------


\newcommand{\dstitle}{Turbulent Ekman flow under stable stratification}
\newcommand{\subtitle}{Set-up and vertical profiles for $Re_D=1000$}
\newcommand{\doi}{10.17169/refubium-43958}

\newcommand{\collectionTitle}{Direct numerical simulation of coupled land–atmosphere systems}
\newcommand{\collectionLink}{\href{https://refubium.fu-berlin.de/handle/fub188/42710}{refubium.fu-berlin.de/handle/fub188/42710}}
\newcommand{\tlabLink}{\href{https://github.com/turbulencia/tlab}{github.com/turbulencia/tlab}}
\newcommand{\tlabName}{\textcolor{gray}{tLab}}
\newcommand{\correspondence}{cedrick@posteo.de}
\newcommand{\authors}{
    Cedrick Ansorge \footnote{\href{mailto:cedrick@posteo.de}{cedrick@posteo.de}}$^\star$, \hspace{0.05cm}  
    Sally Issa $^\star$\hspace{0.05cm},\\
    %Anika Horst $^{\star\star}$ \\[0.5em]
    {\small \light{$^\star$ \emph{ Freie Universit\"at Berlin, Institut f\"ur Meteorologie} \\
   %$^{\star\star}$ \emph{Universit\"at zu K\"oln, Institut f\"ur Geophysik und Meteorologie}
   }
   }
}




\newcommand{\light}[1]{\textcolor{black!80}{#1}}
\allsectionsfont{\bfseries\sffamily} 
\hypersetup{
    colorlinks = true,
    urlcolor   = blue,
    linkcolor  = blue,
    citecolor  = black
}
\newcommand{\printdoi}[1]{\href{https://dx.doi.org/#1}{#1}}

\begin{document}

{ \fontfamily{lmss}\selectfont
\begin{centering}

 \light{\large Dataset description} \\[1em] 
 {\LARGE \bfseries \sffamily\dstitle} \\[1em]
 \light{\large\subtitle} \\[1em]
 \authors\\ 
\end{centering}
\light{\hfill \today}
}


\section{Metadata}
\begin{description}
\sffamily
\item[\textcopyright] This work is licensed under the \href{https://creativecommons.org/licenses/by/4.0}{creative commons CC BY 4.0 license}. \\ {\small\light{You must give appropriate credit, provide a link to the license, and indicate if changes were made . You may do so in any reasonable manner, but not in any way that suggests the licensor endorses you or your use. }}
\item[\sffamily Collection] This data set is part of the collection \href{https://refubium.fu-berlin.de/handle/fub188/42710}{\emph{Turbulent wall-bounded flow}\footnote{\url{refubium.fu-berlin.de/handle/fub188/42710}}}.

{\small \light{The collection is freely available and hosted by Refubium, the  institutional repository of Freie Universit\"at Berlin.}}

\item[\sffamily DOI]\printdoi{\doi}
\item[\sffamily HPC systems] The data was generated under the project \texttt{hku24} on the supercomputer \texttt{JUGENE} at John-von-Neumann Institute for Computing (NIC) at Forschungszentrum J\"ulich (Germany). 
\item[\sffamily Code] The data was generated by the tool-suite for turbulence simulation \href{https://github.com/turbulencia/tlab}{\tlabName\footnote{\url{github.com/turbulencia/tlab}}}

\item[\sffamily Related Publications] \phantom{A}~\\
    \printdoi{10.1007/s10546-014-9941-3}\\
    \printdoi{10.1017/jfm.2016.534} \\
    \printdoi{10.1175/JAS-D-21-0053.1} \\
    \printdoi{10.48550/arXiv.2110.02253} \\
    \printdoi{10.1017/jfm.2018.693}\\
    \printdoi{10.1007/978-3-319-29130-7\_34}\\

\end{description}
\section{The dataset}

\subsection{Contents}
%%% insert description list mentioning each of the contents with a brief description then see, if we still need the text. 

The dataset files are collectively named with grid information and the date of creation of the data on the 
High-Performance Computing (HPC) system. Each file of the collection contains a namelist file (dns.ini), 
which is a plain text file holding the configuration of the tLab code for the respective case 
(for documentation, please refer to Open-source code available under github.com/turbulencia/tlab). \\

\subsection{Physical case}

This dataset contains statistics of resolved-turbulence simulation for turbulent Ekman flow, 
the flow over a flat rotating plate. 
The five simulation cases differ by the strength of stable density stratification imposed as a Dirichlet boundary condition at the surface 
and quantified in the non-dimensional formulation as Froude and Richardson numbers (cf. Tab.~\ref{tab:cases}).
Here, cases are labelled by their Froude numbers as \texttt{fr02, fr05, fr07, fr09, fr18}.

The physical case is characterized by four parameters: geostrophic wind vector \( \vec{G} \), the
fluid kinematic viscosity \( \nu \), the Coriolis parameter \( f \), 
and the buoyancy difference \( B_0 \) between the wall and free stream 
(the molecular Prandtl number equals one, i.e. the kinematic diffusivity equals the viscosity). 
The rossby radius $\Lambda=G/f$ is the length scale implied for this choice of parameters. 
%
We let \( G \equiv \left|\vec{G}\right| \) and align the coordinate direction \( O_x \) with \( \vec{G} \). 
%
The flow is then governed by two dimensionless groups; we choose the Reynolds number Re, the Froude number Fr, and give 
the bulk Richardson number $Ri_B$ (which is a scaled version of the inverse Froude number) for completeness: 
\begin{equation}
    Re_\Lambda = \frac{G \Lambda}{\nu} \quad,\quad
    Fr_\Lambda=\frac{G^2}{B_0 \Lambda} \quad \text{and}\quad
    Ri_{B\text{neutral}} = \frac{B_0 \delta_\text{neutral}}{G^2}.
\end{equation}
Here the bulk Richardson number is evaluated based on the boundary layer thickness of the neutrally stratified 
case such that it does not evolve over the course of simulation. 
The choice of $Re_\Lambda=5\times10^5$ corresponds to an Ekman-based Reynolds-number of 1000 and a friction 
Reynolds number $Re_{\tau} \approx 1400$ for the neutrally stratified case.  All simulations utilize a computational 
grid of 3072 x 512 x 6144 collocation points with a spatial resolution of approximately 
4.1 × 4.1 × 1.05 wall units. The horizontal domain size in terms of the Rossby radius 
\(L_x\) = \(L_y\) = 1.08 \(\Lambda\). 



\begin{table}[H]
    \centering
    \begin{tabular}{lrrrrrr}
        \toprule  
        case identifier&            \texttt{fr18}& \texttt{fr09}& \texttt{fr07}& \texttt{fr05}& \texttt{fr02}\\        
        \bottomrule
        $Re_\Lambda$&                   \multicolumn{5}{c}{$5\times 10^{5}$}\\
        $Fr_{\Lambda}$&             0.1825&	0.0913&	    0.0684&	0.0456&	0.0228\\
        $Ri_{B\text{neutral}}$& 0.29&	0.58&	    0.77&	1.16&	2.32\\
        \hline
        $t_\text{start}/f$&	         9.76&	12.59&	    16.40&	19.29&	23.32\\
        $t_\text{end} /f$&	        12.89&	16.40&      19.29&	21.23&	26.63\\
        $\Delta(t/f)$&	             3.13&	3.81&	    2.90&	1.94&	3.31\\
        \hline
        %$it_\text{start}$&	                237\,600&	256\,050&	    281\,550&	301\,800&	330\,800\\
        %$it_\text{end}$& 	                258\,000&	281\,550&	    301\,750&	315\,750&	356\,000\\
        $\#$iterations&	                20\,400&	25\,500&	    20\,200&	13\,950&	25\,200\\
\toprule            \end{tabular}
    \caption{Simulations used. $t_{\text{start}}/f$ is the time in inertial periods over which the initial 
    condition of the flow is exposed to stable stratification. $\Delta(t/f) = (t_{\text{end}} - t_{\text{start}})/f$  
    is the duration of the simulation and $\#\text{iterations}$ is the number of Runge--Kutta time-integration steps 
    used for integration of the problem over the respective time span.}
    \label{tab:cases}
\end{table} 

\subsection{Variable information}

The statistical data is available in self-documented netCDF format, and it contains a wide array of parameters, encompassing vertical profiles of velocity and scalar variables (temperature/buoyancy as active and for some cases also passive scalars), scalar and momentum budget terms, as well as statistical moments up to the fourth order of velocities, scalars, and derivatives. These parameters provide a comprehensive perspective on Ekman flow dynamics. They are organized into distinct groups. Within the subsequent table, you will find numerous variables grouped together, accompanied by their descriptions and associated equations.

\end{document}





