\documentclass[11pt]{article}
\usepackage{graphicx}
\usepackage[top=10mm, bottom=25mm, left=30mm, right=30mm, includehead]{geometry}
\usepackage{fancyhdr}
\usepackage{hyperref}
\usepackage{mathptmx}
\usepackage{amssymb}
\usepackage{tikz}
\usepackage{longtable}
\usepackage{lipsum}
\usepackage{float}
\usepackage{pdfpages}
\usepackage{sectsty}
\pagestyle{fancy}
\usepackage{lmodern}
\usepackage{enumitem}
\usepackage{multicol}
\usepackage{amsmath}
\usepackage{booktabs}

\fancyhf{}
\date{\nodate}
\setlength{\headheight}{23.10004pt}

\fancyhead[R]{REFUBIUM - FREIE UNIVERSITÄT BERLIN}

% -------------------------------------------------------------
% ------------------ start your abstract here -----------------
% -------------------------------------------------------------

\newcommand{\dstitle}{Turbulent Ekman flow with cubic small-scale surface roughness under stable stratification \\ ($Re_D=1000$, $Ri_\Lambda=[0,\dots,256]$)}
\newcommand{\subtitle}{Direct numerical simulation -- Set-up and vertical profiles}
\newcommand{\doi}{missing}
\newcommand{\collectionTitle}{Turbulent boundary layers simulated by resolved-turbulence simulation}
\newcommand{\collectionLink}{refubium.fu-berlin.de/handle/fub188/42710}
\newcommand{\tlabLink}{github.com/turbulencia/tlab}
\newcommand{\tlabName}{\textcolor{gray}{tLab}}
\newcommand{\correspondence}{j.kostelecky@posteo.de; cedrick@posteo.de}
\newcommand{\authors}{
    Jonathan Kostelecky \footnote{\href{mailto:j.kostelecky@posteo.de}{j.kostelecky@posteo.de}}$^{\star,\blacktriangle}$, \hspace{0.05cm}  
    Cedrick Ansorge     \footnote{\href{mailto:cedrick@posteo.de}{cedrick@posteo.de}}$^\star$ \\  
    {\small \light{$^\star$ \emph{ Freie Universit\"at Berlin, Institut f\"ur Meteorologie} \\
    $^{\blacktriangle}$ \emph{Universit\"at zu K\"oln, Institut f\"ur Geophysik und Meteorologie}}}}
\newcommand{\printdoi}[1]{\href{https://dx.doi.org/#1}{#1}}
\newcommand{\light}[1]{\textcolor{black!80}{#1}}
\allsectionsfont{\bfseries\sffamily} 
\hypersetup{
    colorlinks = true,
    urlcolor   = blue,
    linkcolor  = blue,
    citecolor  = black
}
%-----------------------------------------------------------------------------------------------------------------------------------
%-----------------------------------------------------------------------------------------------------------------------------------
\begin{document}
%-----------------------------------------------------------------------------------------------------------------------------------
%-----------------------------------------------------------------------------------------------------------------------------------

{\fontfamily{lmss}\selectfont
\begin{centering}
    \light{\large Dataset description} \\[1em] 
    {\LARGE \bfseries \sffamily\dstitle} \\[1em]
    \light{\large\subtitle} \\[1em]
    \authors \\ 
\end{centering}
\light{\hfill\today}}

%-----------------------------------------------------------------------------------------------------------------------------------
\section{Metadata}
%-----------------------------------------------------------------------------------------------------------------------------------

\begin{description}

    \sffamily

\item[\textcopyright] This work is licensed under the \href{https://creativecommons.org/licenses/by/4.0}{creative commons CC BY 4.0 license}. \\ 
{\small\light{You must give appropriate credit, provide a link to the license, and indicate if changes were made. You may do so in any reasonable manner, but not in any way that suggests the licensor endorses you or your use. }}

\item[\sffamily Collection] This dataset is part of the collection \href{https://refubium.fu-berlin.de/handle/fub188/42710}{\emph{Turbulent wall-bounded flow}\footnote{\url{refubium.fu-berlin.de/handle/fub188/42710}}}. \\ 
{\small \light{The collection is freely available and hosted by Refubium, the  institutional repository of Freie Universit\"at Berlin.}}

\item[\sffamily DOI]\printdoi{\doi}

\item[\sffamily HPC systems] The data was generated under the project \texttt{TrainABL} on the supercomputer \texttt{HAWK} at Höchstleistungsrechenzentrum Stuttgart (HLRS, in Germany). 

\item[\sffamily Code] The data was generated by the tool-suite for turbulence simulation \href{https://github.com/turbulencia/tlab}{\tlabName\footnote{\url{github.com/turbulencia/tlab}}}.

\item[\sffamily Related Publication] \phantom{A}~\\
    \printdoi{10.1017/jfm.2024.542}\\

\end{description}

%-----------------------------------------------------------------------------------------------------------------------------------
\section{The dataset}
%-----------------------------------------------------------------------------------------------------------------------------------
\subsection{Contents}
%-----------------------------------------------------------------------------------------------------------------------------------
The dataset files are collectively named with grid information and the date of creation of the data on the High-Performance Computing (HPC) system. Each file of the collection contains time-series of statistical data for flow and scalar variables in a self-documented netCDF format and a namelist file (dns.ini), which is a plain text file holding the configuration of the tLab code for the respective case (for documentation, please refer to the open-source code available under \url{github.com/turbulencia/tlab}). In the case of heterogeneous surface conditions, an additional netCDF file is provided that describes the geometry of the surface roughness.

%-----------------------------------------------------------------------------------------------------------------------------------
\subsection{Physical case}
%-----------------------------------------------------------------------------------------------------------------------------------
% N, S001, S002, S004, S005, S008, S012, S016, S020, S032, S042, S064, S128, S128P, S192P, S256P
This dataset contains 16 simulation cases (\texttt{ID: N, S001,\dots, S128, \dots, S256P}; labelling: numbering according to the value of $Ri_\Lambda$, \texttt{N} for neutral, \texttt{S} for stable, \texttt{P} for concurrent runs), which are driven by the same large-scale forcing and an increasing strength of stable density stratification imposed as Dirichlet boundary condition (viz. constant temperature difference between upper and lower domain boundary). The simulations were carried out sequentially in time starting from the neutral case \texttt{N}, only the most stable cases (\texttt{S128P, S192P, S256P}) were carried out in parallel (cf. overview of the simulation cases in Table~\ref{tab:cases}). 
The small-scale surface roughness at the lower domain boundary is identical for all cases and consists of $56^2$ square blocks with a uniform height and width distribution and features the same statistical properties as case \texttt{r3} in \printdoi{10.1017/jfm.2024.542}.\\
%
The physical case is characterized by four parameters: the geostrophic wind vector \linebreak $\mathbf{G}=\left(G_1,G_2,0\right)^T$ (with the magnitude $G=\sqrt{G_1^2+G_2^2}$, rotated here by $\approx18.1^\circ$ w.r.t. the coordinate direction $O_x$), the constant kinematic fluid viscosity $\nu$, the Coriolis parameter $f$, and the buoyancy difference $B_0$ between the wall and free stream. The Rossby radius $\Lambda=G/f$ is the length scale implied for this choice of parameters. The flow is then governed by two dimensionless numbers, since the molecular Prandtl number $Pr=1$, i.e. the kinematic diffusivity equals the viscosity, and the Rossby number $Ro=1$. The Reynolds number $Re$ and the Froude number $Fr$ are given as (for comparison, the scaled bulk Richardson number $Ri_B$) 
%
\begin{equation}
    Re_\Lambda = \frac{G \Lambda}{\nu},\quad
    Fr_\Lambda=\frac{G^2}{B_0 \Lambda}=Ri_\Lambda^{-1}\quad \mathrm{and} \quad
    Ri_{B} = \frac{B_0 \delta_\mathrm{neutral}}{G^2}.
\end{equation}
%
Here the bulk Richardson number $Ri_B$ is evaluated based on the boundary layer thickness of the neutrally stratified case \texttt{N} such that it does not evolve over the course of simulation. The choice of $Re_\Lambda=5\cdot10^5$ corresponds to $Re_D=1000$ ($Re_D=DG/\nu$, with the laminar Ekman-layer depth $D=\sqrt{2\nu/ f}$) and a friction Reynolds number $Re_{\tau} \approx 2700$ for the neutrally stratified case.  
%
Two computational grids are used: (A) $N_{xz}\times N_y=3840^2\times704$, (B) $3840^2\times576$ with similar spatial resolution of approximately $\Delta xz_{N}^+\times\Delta y_{N,min}^+=2.6^2\times1.0$ wall units and a domain size in terms of the Rossby radius (A)~$(L_{xz}\times L_y)/\Lambda^3=0.27^2\times0.26$, (B)~$0.27^2\times0.11$. 
%
%-----------------------------------------------------------------------------------------------------------------------------------
\subsection{Variable information} 
%-----------------------------------------------------------------------------------------------------------------------------------
The statistical data is available in self-documented netCDF format, and it contains a wide array of parameters, encompassing vertical profiles of velocity and scalar variables (temperature/buoyancy as active and for some cases also passive scalars), scalar and momentum budget terms, as well as statistical moments up to the fourth order of velocities, scalars, and derivatives. These parameters provide a comprehensive perspective on Ekman flow dynamics. They are organized into distinct groups. Within the subsequent table, you will find numerous variables grouped together, accompanied by their descriptions and associated equations. \\
In order to fully describe the geometry of the surface roughness, there is a horizontal domain slice in netCDF format, that describe the positions and heights of the roughness elements in grid points.
\par
%-----------------------------------------------------------------------------------------------------------------------------------
\begin{table}[H]
    \centering
    \begin{tabular}{lrcccccr}
        Case (\texttt{ID}) & $Ri_\Lambda$ & $Ri_B$ & Grid &  $t^-_\text{start} [1/f]$ &  $t^-_\text{end} [1/f]$ &  $\Delta t^-_\text{sim} [1/f]$ &  \#iterations \\ \bottomrule
        \texttt{N    } &   0 & \hphantom{1}0.000 & A & 64.96 & 66.02 & 1.06 & 29500  \\ [1ex]
        \texttt{S001 } &   1 & \hphantom{1}0.073 & A & 66.02 & 66.61 & 0.58 & 16000  \\
        \texttt{S002 } &   2 & \hphantom{1}0.147 & A & 66.61 & 67.06 & 0.44 & 11800  \\
        \texttt{S004 } &   4 & \hphantom{1}0.293 & A & 67.06 & 67.78 & 0.71 & 17800  \\
        \texttt{S005 } &   5 & \hphantom{1}0.367 & A & 67.78 & 68.26 & 0.48 & 11800  \\
        \texttt{S008 } &   8 & \hphantom{1}0.587 & A & 68.26 & 69.02 & 0.74 & 17800  \\
        \texttt{S012 } &  12 & \hphantom{1}0.880 & A & 69.02 & 69.53 & 0.51 & 11800  \\
        \texttt{S016 } &  16 & \hphantom{1}1.174 & A & 69.53 & 69.98 & 0.44 & 10300  \\
        \texttt{S020 } &  20 & \hphantom{1}1.467 & A & 69.98 & 70.38 & 0.39 & 8800  \\
        \texttt{S032 } &  32 & \hphantom{1}2.348 & A & 70.38 & 70.84 & 0.45 & 9800  \\
        \texttt{S042 } &  42 & \hphantom{1}3.081 & A & 70.84 & 71.40 & 0.55 & 11800  \\
        \texttt{S064 } &  64 & \hphantom{1}4.695 & A & 71.40 & 71.84 & 0.43 & 8800  \\ 
        \texttt{S128 } & 128 & \hphantom{1}9.390 & A & 71.84 & 72.32 & 0.48 & 8800  \\ [1ex] 
        \texttt{S128P} & 128 & \hphantom{1}9.390 & B & 72.32 & 74.27 & 1.95 & 40600  \\
        \texttt{S192P} & 192 & 14.086            & B & 73.49 & 74.46 & 0.97 & 19000  \\
        \texttt{S256P} & 256 & 18.781            & B & 73.49 & 75.68 & 2.19 & 41400  \\
    \end{tabular}
    \caption{Simulations cases of this dataset. Time in eddy-turnover times $f^{-1}$; $t^-_\text{start}$ start, $t^-_\text{end}$ end and $\Delta t^-_\text{sim}$ total simulation time of the cases. $\#\text{iterations}$ is the total number of Runge--Kutta time-integration steps used for integration of the problem over the respective time span.}
    \label{tab:cases}
\end{table} 
%-----------------------------------------------------------------------------------------------------------------------------------
%-----------------------------------------------------------------------------------------------------------------------------------
\end{document}
%-----------------------------------------------------------------------------------------------------------------------------------
%-----------------------------------------------------------------------------------------------------------------------------------