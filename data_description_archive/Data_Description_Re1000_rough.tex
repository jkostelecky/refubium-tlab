\documentclass[11pt]{article}
\usepackage{graphicx}
\usepackage[top=10mm, bottom=25mm, left=30mm, right=30mm, includehead]{geometry}
\usepackage{fancyhdr}
\usepackage{hyperref}
\usepackage{mathptmx}
\usepackage{amssymb}
\usepackage{tikz}
\usepackage{longtable}
\usepackage{lipsum}
\usepackage{float}
\usepackage{pdfpages}
\usepackage{sectsty}
\pagestyle{fancy}
\usepackage{lmodern}
\usepackage{enumitem}
\fancyhf{}
\date{\nodate}
\setlength{\headheight}{23.10004pt}

\fancyhead[R]{REFUBIUM - FREIE UNIVERSITÄT BERLIN}

% -------------------------------------------------------------
% ------------------ start your abstract here -----------------
% -------------------------------------------------------------


\newcommand{\dstitle}{Turbulent Ekman flow ($Re_D=1000$, $Ri=0$)}
\newcommand{\subtitle}{Direct numerical simulation -- Set-up and vertical profiles}
\newcommand{\doi}{10.17169/refubium-43215}

\newcommand{\collectionTitle}{Turbulent boundary layers simulated by resolved-turbulence simulation}
\newcommand{\collectionLink}{refubium.fu-berlin.de/handle/fub188/42710}
\newcommand{\tlabLink}{github.com/turbulencia/tlab}
\newcommand{\tlabName}{\textcolor{gray}{tLab}}
\newcommand{\correspondence}{j.kostelecky@posteo.de; cedrick@posteo.de}
\newcommand{\authors}{
    Jonathan Kostelecky \footnote{\href{mailto:j.kostelecky@posteo.de}{j.kostelecky@posteo.de}}$^{\star,\blacktriangle}$, \hspace{0.05cm}  
    Cedrick Ansorge     \footnote{\href{mailto:cedrick@posteo.de}{cedrick@posteo.de}}$^\star$ \\  
    {\small \light{$^\star$ \emph{ Freie Universit\"at Berlin, Institut f\"ur Meteorologie} \\
    $^{\blacktriangle}$ \emph{Universit\"at zu K\"oln, Institut f\"ur Geophysik und Meteorologie}}}}

\newcommand{\light}[1]{\textcolor{black!80}{#1}}
\allsectionsfont{\bfseries\sffamily} 
\hypersetup{
    colorlinks = true,
    urlcolor   = blue,
    linkcolor  = blue,
    citecolor  = black
}
\newcommand{\printdoi}[1]{\href{https://dx.doi.org/#1}{#1}}

\begin{document}

{ \fontfamily{lmss}\selectfont
\begin{centering}

 \light{\large Dataset description} \\[1em] 
 {\LARGE \bfseries \sffamily\dstitle} \\[1em]
 \light{\large\subtitle} \\[1em]
 \authors\\ 
\end{centering}
\light{\hfill \today}
}


\section{Metadata}
\begin{description}
\sffamily
\item[\textcopyright] This work is licensed under the \href{https://creativecommons.org/licenses/by/4.0}{creative commons CC BY 4.0 license}. \\ {\small\light{You must give appropriate credit, provide a link to the license, and indicate if changes were made. You may do so in any reasonable manner, but not in any way that suggests the licensor endorses you or your use. }}
\item[\sffamily Collection] This dataset is part of the collection \href{https://refubium.fu-berlin.de/handle/fub188/42710}{\emph{Turbulent wall-bounded flow}\footnote{\url{refubium.fu-berlin.de/handle/fub188/42710}}}.

{\small \light{The collection is freely available and hosted by Refubium, the  institutional repository of Freie Universit\"at Berlin.}}

\item[\sffamily DOI]\printdoi{\doi}
\item[\sffamily HPC systems] The data was generated under the project \texttt{TrainABL} on the supercomputer \texttt{HAWK} at Höchstleistungsrechenzentrum Stuttgart (HLRS, in Germany). 
\item[\sffamily Code] The data was generated by the tool-suite for turbulence simulation \href{https://github.com/turbulencia/tlab}{\tlabName\footnote{\url{github.com/turbulencia/tlab}}}.
%\item[\sffamily Related Publications] ~\\
    %\printdoi{xxxxxxx} \\
\end{description}

\section{The dataset}

\subsection{Contents}
The dataset files, collectively named with grid information and the date of creation of the data on the High-Performance Computing (HPC) system. 
Each file of the collection contains time-series of statistical data for flow and scalar variables in a self-documented netCDF format and a namelist with the file name dns.ini, which is a plain text file holding the configuration of the tLab code (for documentation, please refer to the open-source code available under \url{github.com/turbulencia/tlab}). In the case of heterogeneous surface conditions, an additional netCDF file is provided that describes the geometry of the surface roughness.

\subsection{Physical case}

This dataset contains 4 simulation cases (\texttt{ID: s, r1, r2, r3}), with a similar computational grid, domain size and driven by the same large-scale forcing, but differ in the surface condition. Case \texttt{s} has a smooth surface and the rough cases \texttt{r1, r2, r3} feature each $56^2$ square blocks on the lower domain boundary with a uniform height and width distribution. The mean height of the roughness elements increases from case \texttt{r1} via \texttt{r2} to \texttt{r3}.
These simulation cases, conducted with a Reynolds number of $Re_D=1000$ ($Re_D=DG/\nu$, with the laminar Ekman-layer depth $D=\sqrt{2\nu/ f}$, Coriolis parameter $f$, geostrophic wind $G$ and the kinematic viscosity $\nu$), corresponding to a friction Reynolds number \(Re_{\tau}\) of 1408 for the smooth case \texttt{s}, delves into the study of the turbulent flow with small-scale surface roughness. Utilizing a computational grid measuring 3072 x 656 x 3072 collocation points with a spatial resolution of 2.3 x 1.0 x 2.3 wall units (smooth case), the domain size is scaled to \(L_x\) = \(L_z\) = 0.27 \(\Lambda\), where \(\Lambda=G/f\) is the Rossby radius.\\

\subsection{Variable information} 

The statistical data is available in self-documented netCDF format, and it contains a wide array of parameters, encompassing vertical profiles of velocity and scalar variables (temperature/buoyancy as active and for some cases also passive scalars), scalar and momentum budget terms, as well as statistical moments up to the fourth order of velocities, scalars, and derivatives. These parameters provide a comprehensive perspective on Ekman flow dynamics. They are organized into distinct groups. Within the subsequent table, you will find numerous variables grouped together, accompanied by their descriptions and associated equations. \\
In order to fully describe the geometry of the surface roughness, there are horizontal domain slices for each of the rough cases (\texttt{r1, r2, r3}) in netCDF format, that describe the positions and heights of the roughness elements in grid points.

\end{document}