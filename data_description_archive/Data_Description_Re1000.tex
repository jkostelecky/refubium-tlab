\documentclass[11pt]{article}
\usepackage{graphicx}
\usepackage[top=10mm, bottom=25mm, left=30mm, right=30mm, includehead]{geometry}
\usepackage{fancyhdr}
\usepackage{hyperref}
\usepackage{mathptmx}
\usepackage{tikz}
\usepackage{longtable}
\usepackage{lipsum}
\usepackage{float}
\usepackage{pdfpages}
\usepackage{sectsty}
\pagestyle{fancy}
\usepackage{lmodern}
\usepackage{enumitem}
\fancyhf{}
\date{\nodate}
\setlength{\headheight}{23.10004pt}

\fancyhead[R]{REFUBIUM - FREIE UNIVERSITÄT BERLIN}

% -------------------------------------------------------------
% ------------------ start your abstract here -----------------
% -------------------------------------------------------------


\newcommand{\dstitle}{Turbulent Ekman flow ($Re_D=1000$, $Ri=0$)}
\newcommand{\subtitle}{Direct numerical simulation -- Set-up and vertical profiles}
\newcommand{\doi}{10.17169/refubium-42507}

\newcommand{\collectionTitle}{Turbulent boundary layers simulated by resolved-turbulence simulation}
\newcommand{\collectionLink}{refubium.fu-berlin.de/handle/fub188/42710}
\newcommand{\tlabLink}{github.com/turbulencia/tlab}
\newcommand{\tlabName}{\textcolor{gray}{tLab}}
\newcommand{\correspondence}{cedrick@posteo.de}
\newcommand{\authors}{
    Cedrick Ansorge \footnote{\href{mailto:cedrick@posteo.de}{cedrick@posteo.de}}$^\star$, \hspace{0.05cm}  
    Sally Issa $^\star$\hspace{0.05cm},
    Anika Horst $^{\star\star}$ \\[0.5em]
    {\small \light{$^\star$ \emph{ Freie Universit\"at Berlin, Institut f\"ur Meteorologie} \\
    $^{\star\star}$ \emph{Universit\"at zu K\"oln, Institut f\"ur Geophysik und Meteorologie}}}
}




\newcommand{\light}[1]{\textcolor{black!80}{#1}}
\allsectionsfont{\bfseries\sffamily} 
\hypersetup{
    colorlinks = true,
    urlcolor   = blue,
    linkcolor  = blue,
    citecolor  = black
}
\newcommand{\printdoi}[1]{\href{https://dx.doi.org/#1}{#1}}

\begin{document}

{ \fontfamily{lmss}\selectfont
\begin{centering}

 \light{\large Dataset description} \\[1em] 
 {\LARGE \bfseries \sffamily\dstitle} \\[1em]
 \light{\large\subtitle} \\[1em]
 \authors\\ 
\end{centering}
\light{\hfill \today}
}


\section{Metadata}
\begin{description}
\sffamily
\item[\textcopyright] This work is licensed under the \href{https://creativecommons.org/licenses/by/4.0}{creative commons CC BY 4.0 license}. \\ {\small\light{You must give appropriate credit, provide a link to the license, and indicate if changes were made. You may do so in any reasonable manner, but not in any way that suggests the licensor endorses you or your use. }}
\item[\sffamily Collection] This dataset is part of the collection \href{https://refubium.fu-berlin.de/handle/fub188/42710}{\emph{Turbulent wall-bounded flow}\footnote{\url{refubium.fu-berlin.de/handle/fub188/42710}}}.

{\small \light{The collection is freely available and hosted by Refubium, the  institutional repository of Freie Universit\"at Berlin.}}


\item[\sffamily DOI]\printdoi{\doi}
\item[\sffamily HPC systems] The data was generated under the project \texttt{HKU24} on the supercomputer \texttt{JUQUEEN} at John-von-Neumann Institute for Computing (NIC) at Forschungszentrum J\"ulich (Germany). 
\item[\sffamily Code] The data was generated by the tool-suite for turbulence simulation \href{https://github.com/turbulencia/tlab}{\tlabName\footnote{\url{github.com/turbulencia/tlab}}}

\item[\sffamily Related Publications] ~\\
    \printdoi{10.1017/jfm.2018.693} \\
    \printdoi{10.1007/s10546-014-9941-3} \\
    \printdoi{10.1007/s10546-018-0386-y} \\
    \printdoi{10.1017/jfm.2016.534} \\
    \printdoi{10.1175/JAS-D-21-0053.1}\\

\end{description}
\section{The dataset}

\subsection{Contents}
%%% insert description list mentioning each of the contents with a brief description then see, if we still need the text. 

The dataset files, collectively named with grid information and the date of creation of the data on the High-Performance Computing (HPC) system. Each file of the collection contains time-series of a namelist files name dns.ini which is a plain text file holding the configuration of the tLab code (for documentation, please refer to open-source code available under github.com/turbulencia/tlab). \\

\subsection{Physical case}

This case of simulation conducted with a Reynolds number of $Re_D=1000$ ($Re_D=DG/\nu$, with the laminar Ekman-layer depth $D=\sqrt{2\nu/ f}$, Coriolis parameter $f$, geostrophic wind $G$ and the kinematic viscosity $\nu$), corresponding to a friction Reynolds number \(Re_{\tau}\) of 1403, delves into the study of the turbulent flow. Utilizing a computational grid measuring 3072 x 512 x 6144 collocation points with a spatial resolution of 9.3 x 1.14 x 4.7  wall units, the domain size is scaled to \(L_x\) = \(L_z\) = 1.08 \(\Lambda\), where \(\Lambda=G/f\) is the Rossby radius.\\

\subsection{Variable information}

The statistical data is available in self-documented netCDF format, and it contains a wide array of parameters, encompassing vertical profiles of velocity and scalar variables (temperature/buoyancy as active and for some cases also passive scalars), scalar and momentum budget terms, as well as statistical moments up to the fourth order of velocities, scalars, and derivatives. These parameters provide a comprehensive perspective on Ekman flow dynamics. They are organized into distinct groups. Within the subsequent table, you will find numerous variables grouped together, accompanied by their descriptions and associated equations.


\end{document}